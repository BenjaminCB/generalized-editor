This project was exploratory in nature, aiming to implement a generic editor based on the work of the Aalborg project\cite{aalborg} and related work. The primary objective was to create a type-safe, generalized syntax-directed editor capable of supporting any language, including but not limited to programming languages. The implementation achieved a minimal viable product that can instantiate a syntax-directed editor for any language, with basic editing functionalities such as cursor movement and substitutions. This was demonstrated through practical examples involving subsets of different languages, including C, SQL, and LaTeX.

Only atomic prefix commands have been implemented at the time of writing.
This however serves as a necessary and big first step toward completing the remaining editor expressions.

The implementation does not handle context-sensitive syntax, specifically
binding mechanisms, at the time of writing. However, the model in the implementation has been designed to hold information about binders, and
the specification language supports binders.

The implementation also lacks views of the code being edited. For now,
the editor can only be interacted with through the Elm REPL, as shown in the examples.

Ultimately, the specification language enables the user to define the syntax (both abstract and concrete) of any language
in a non-challenging way, assuming the user is familiar with syntax and language constructs in general.

\section{Future work}
Editor expressions for the remaining commands should be implemented.This would allow for a more complete editor, capable of handling all editor expression provided by the calculus. This is possible by following a similar approach to the one used for the atomic prefix commands, i.e. by implementing function generators for each command based on the semantics provided in the editor calculus.

Different views is also a feature that should be implemented, as this also
was a criteria for a good solution. This would allow the editor to display the syntax tree in different ways, such as a tree view or string-format (pretty-printing). The implementation of the non-generic
structure editor\cite{KU-bach} was successful in having a tree view, which can
serve as inspiration. The model in this implementation is already designed to hold the concrete syntax of an operator (the \texttt{term} field in the \texttt{Syntax} record), allowing for pretty-printing to be implemented with relative ease.

There is currently no active use of binders in the editor, leaving
out the possibility of potentially checking for use-before-declarations or
shadowing. Implementing this is a non-trivial task, as it requires a more
complex model of the syntax tree, as well as a more complex implementation of the editor expressions. Implementing this is would bring the implementation
closer to being a complete realization of the editor calculus.

Given the generic nature of the editor, it would be interesting to see
an implementation in a language with support for typeclasses, such as Haskell.
This should allow for a more concise implementation overall, being less rigid
than the current implementation, since it relies on references to functions and
data types via a low-level interface in the Elm CodeGen package, which refers
to them with manipulated strings. This is in contrast to typeclass
instances, which would allow for a more abstract representation, with editor functions supporting any tree of any sort, without the need for explicit references to functions implementing actions for each sort.

As the last point of future work, a minor improvement to the specification
language would be to add support for defining the starting symbol of the
language. This would shrink the amount of code in the
source code generator, but also significantly reduce the amount of generated lines of code, since this would eliminate the need for supporting all
sorts in some cases.
